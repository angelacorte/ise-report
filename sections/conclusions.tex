\section{Conclusions}
The goal of this project was to give users of the \textit{framework} JaKtA the possibility to create distributed multi-agent systems easily and quickly, with the least possible \textit{overhead}.

Therefore, an extension of the \textit{framework} was developed, providing a new implementation of the multi-agent system concept capable of communicating with other systems using an asynchronous \textit{publish-subscribe} communication model.

This communication model was implemented through the \textit{framework} Ktor and the \textit{WebSocket} protocol.

The team believes that the goal has been achieved; however, they are aware of some possible improvements to the proposed solution, which will be discussed in the next section.

In conclusion, the project has provided a good opportunity to deepen the knowledge acquired in the field of multi-agent systems, agent-oriented programming, and distributed systems.


\subsection{Future Work}
The team believes that the project has achieved its set objectives; however, it is aware that the proposed solution can be improved in various aspects, some of which are listed below:
\begin{itemize}
    \item \textbf{Support for Multiple Communication Protocols}: Currently, the \textit{framework} supports only one communication protocol, namely \textit{WebSocket}. However, it would be interesting
    to provide users with the option to choose from different protocols, such as \textit{MQTT}.
    The project's architecture would easily allow adding \textit{MQTT} support through the implementation of a new \texttt{Network} and potentially a new \textit{broker}.
    \item \textbf{Handling Uniqueness of Remote Service Names}: As mentioned in the "Implementation Details" section, the \textit{framework} does not provide any mechanism to ensure
    the uniqueness of remote service names.
    One possible solution could be to add a coordination step during the connection to the \textit{broker}, allowing \texttt{Dmas} to register their services and check that the chosen name has not been used already.
\end{itemize}
