\section{Conclusioni}
L'obiettivo di questo progetto era dare la possibilità agli utenti del framework JaKTa la possibilità di creare sistemi multi-agente distribuiti in modo semplice e veloce, 
con il minore overhead possibile.

È stata quindi sviluppata un'estensione del framework che fornisce una nuova implementazione del concetto di sistema multi-agente, in grado di comunicare con gli altri sistemi
con modello di comunicazione asincrono publish-subscribe.

Questo modello di comunicazione è stato implementato attraverso il framework Ktor ed il protocollo Websocket. \\

Il gruppo ritiene che l'obiettivo sia stato raggiunto, tuttavia è conscio di alcuni possibili miglioramenti alla soluzione proposta, che verranno discussi nella sezione successiva.\\

In conclusione, il progetto ha costituito una buona occasione per approfondire le conoscenze acquisite nell'ambito dei sistemi multi-agente, della programmazione agent-oriented e dei sistemi distribuiti.


\subsection{Sviluppi Futuri}

Recap what you did \emph{not}

