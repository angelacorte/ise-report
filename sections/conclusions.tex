\section{Conclusioni}
L'obiettivo di questo progetto era dare la possibilità agli utenti del \textit{framework} JaKtA la possibilità di creare sistemi multi-agente distribuiti in modo semplice e veloce, con il minore \textit{overhead} possibile.

È stata quindi sviluppata un'estensione del \textit{framework} che fornisce una nuova implementazione del concetto di sistema multi-agente, in grado di comunicare con gli altri sistemi con modello di comunicazione asincrono \textit{publish-subscribe}.

Questo modello di comunicazione è stato implementato attraverso il \textit{framework} Ktor ed il protocollo \textit{WebSocket}.

Il team ritiene che l'obiettivo sia stato raggiunto, tuttavia è conscio di alcuni possibili miglioramenti alla soluzione proposta, che verranno discussi nella sezione successiva.

In conclusione, il progetto ha costituito una buona occasione per approfondire le conoscenze acquisite nell'ambito dei sistemi multi-agente, della programmazione \textit{agent-oriented} e dei sistemi distribuiti.


\subsection{Sviluppi Futuri}
Il team ritiene che il progetto abbia raggiunto gli obiettivi prefissati, tuttavia è consapevole che la soluzione proposta possa essere migliorata in diversi aspetti, alcuni dei quali vengono elencati di seguito:

\begin{itemize}
    \item \textbf{Supporto a più protocolli di comunicazione}: attualmente il \textit{framework} supporta un solo protocollo di comunicazione, ovvero \textit{WebSocket}. Tuttavia sarebbe interessante
    fornire agli utenti la possibilità di scegliere tra diversi protocolli, come ad esempio \textit{MQTT}. L'architettura del progetto permetterebbe agevolmente l'aggiunta del supporto ad \textit{MQTT}
    tramite l'implementazione di una nuova \texttt{Network} ed eventualmente un nuovo \textit{broker}.
    \item \textbf{Gestione dell'univocità dei nomi dei servizi remoti}: come già accennato nella sezione "Dettagli Implementativi", il \textit{framework} non fornisce alcun meccanismo per garantire
    l'univocità dei nomi dei servizi remoti. Una possibile soluzione può essere quella di aggiungere uno step di coordinazione al momento della connessione al \textit{broker}, in modo da dare la possibilità ai
    \texttt{Dmas} di registrare i propri servizi e verificare che il nome scelto non sia già stato utilizzato.
\end{itemize}
