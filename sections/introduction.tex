\section{Introduction}

\subsection{Agents}

\subsubsection{Definition}
An agent is an autonomous computational entity endowed with decision-making and independent execution capabilities. Agents are designed to operate in a specific environment, assuming defined responsibilities and tasks. The key characteristic of an agent is its autonomy, which means it can make decisions independently, without requiring external intervention for every action undertaken. This computational autonomy implies intrinsic proactivity, wherein agents don't passively await events but actively act to change the state of the Multi-Agent System (MAS).

A MAS, or Multi-Agent System, is a computational system composed of multiple autonomous agents that interact with each other to achieve common or individual goals. In a MAS, each agent is an autonomous entity with perception, reasoning, decision-making, and action capabilities. Agents in a MAS can operate independently and, simultaneously, collaborate to address complex problems or achieve goals that may be difficult or impossible for a single agent to accomplish.

A crucial aspect is the situational nature of agents, meaning their actions depend on the context in which they find themselves. Agents, by interacting with each other and the surrounding environment, become inherently social entities. This sociality emerges because autonomy makes sense only when an agent is immersed in a society of agents, highlighting the absence of truly autonomous agents in isolation. Interaction occurs through the exchange of knowledge and information, while the proactivity of agents drives them to change the state of the MAS, influencing other agents or the environment. In this context, agents become autonomous, interactive, and social components that, through their self-regulation capabilities, are essential for managing complex systems.

\subsubsection{The Framework Belief-Desire-Intention}
The Belief-Desire-Intention (BDI) framework is widely accepted as one of the most popular and successful frameworks in the field of agent technology. Defined by Rao and Georgeff, the framework is based on three key concepts: belief, desire, and intention. Agents that adhere to this framework are commonly referred to as BDI agents.
The following are the characteristics of the framework:
\begin{itemize}
    \item \textbf{Belief}: To make decisions, agents must have a representation of the world in which they exist. Beliefs are the knowledge about the environment that agents gather and can store as part of their belief base.
    \item \textbf{Desire}: Desires represent the main goals of the system. Agents have a set of plans at their disposal that they can use to achieve their goals, and the choice of the most suitable plan depends on the current situation of the agent and the environment.
    \item \textbf{Intention}: Intentions capture the deliberative component of the system. They represent the actions to which the agent has committed to executing to achieve its goals. Intentions keep track of the progress of actions and changes in the environment during execution.
\end{itemize}

\subsection{JaKtA}
JaKtA\cite{10.1007/978-3-031-43264-4_4} is a library that provides the capability to develop agents that perceive and act within a shared environment and communicate with each other through a message-based communication system.

The library offers a framework for building multi-agent systems (MAS) composed of agents conforming to the Belief-Desire-Intention (BDI) model and inspired by the implementation of Jason\cite{Bordini2005}. A Multi-Agent System consists of two fundamental entities:

\begin{itemize}
    \item \textbf{Agents}: Agents are the main entities of the library.
    \item \textbf{Environment}: The environment in which agents live.
\end{itemize}

During their execution, agents in a MAS autonomously observe the environment's state and choose to act based on perceived information. Their actions can influence the state of the environment, and this change will be perceived by all other agents operating on it. Agents are equipped with goals that they are committed to achieving, contributing to the overall goal of the Multi-Agent System. As for the environment, a user can:

\begin{itemize}
    \item Create an environment.
    \item Add functionalities to the environment, such as:
    \begin{itemize}
        \item Define the actions that agents within the environment can use.
        \item Define the state of the environment, including the set of information that agents can perceive and potentially modify.
    \end{itemize}
    \item Add and remove agents from the environment.
\end{itemize}

Moreover, agents located in the same environment can communicate with each other through messages. Messages allow agents to share information about their state or share the completion of a goal, making the Multi-Agent System a true society of agents. Regarding agents, a user can:

\begin{itemize}
    \item Create and customize agents' behavior, following the BDI model.
    \item Add actions related to an agent.
    \item Customize the behavior of agents during their execution.
\end{itemize}

The library enables users to define how their BDI agents will be executed. This means that the same agents can be executed either in a single thread or among multiple threads, in a nearly transparent manner. To enable the MAS to do this, users must define its execution model, i.e., how its agents are executed.

\subsection{Agent Life Cycle in JaKtA}

The life cycle of an agent essentially serves as the mechanism responsible for managing the execution of the entity. Indeed, at each iteration of the latter, a series of operations must be performed to guarantee the functioning of its BDI model. The first operation that the agent must perform, at each iteration of its life cycle, is to observe whether any changes have occurred within the environment. Indeed, to make the agent as responsive as possible to these changes, before it decides which operations to perform, it must observe the current state of the environment. Once the agent knows the new state of the environment, it must update its knowledge base, i.e. the belief base, with the new information retrieved. To react to changes in the environment's state, and therefore to be a reactive entity, the agent must generate events. These events are determined by evaluating differences between its knowledge base and the new perception of the state around itself. Events can be of two types: addition or deletion of a belief. Another event that the agent must respond to is the receipt of a new message. Messages are the components that agents can use to communicate with each other within the environment. This means that, at each iteration of the life cycle, the Agent has to check if there are any new messages that it needs to handle. Messages can trigger two different scenarios related to their type, indeed, there could be \textit{achieve messages} that trigger a new \textit{achievement goal invocation event} or also \textit{tell messages} that are handled exactly like adding a new belief, the only difference with a generic one of them is given by the annotation, because it refers to the sender of that message. As with the perception phase, the latter type of message also generates a \textit{belief addition event}. After checking all the event sources that have occurred, the agent selects one event to handle. If there is an event to handle, then it is removed from the agent's event set, and a suitable plan is searched to fulfill it. The plan selection phase follows three steps.

\begin{itemize}
    \item First, the plans that have a triggering event that unifies with the selected event are filtered among plans available to the agent, these are the relevant plans, and then the applicable ones are selected; this is done by verifying that the plan's context is a logical consequence of the agent's belief base.
    \item Of the remaining plans, the one applied is chosen. If it is possible to outline a plan that satisfies all these three steps, then this is assigned to an intention. If the event that triggered the plan application was external, then a new intention, with the partially instantiated goals of that plan's body, is added to the intention set. Otherwise, if the event was an internal one, the life cycle retrieves the related intention and adds another activation record, with those goals inside of it, on the top of its stack.
    \item Lastly, the agent executes one intention goal. The intention to execute is selected and if there's at least one of them to execute then it's scheduled for execution. The execution phase is where the agent could potentially perform changes over its state or on the environment's one.
\end{itemize}

However, agent life cycle does not operate directly over the environment state, as for action's side effect, the whole reasoning procedure described above returns a list of environment changes, which application will be managed by the Multi-Agent System execution strategy.

\subsection{Environment in JaKtA}

The \textit{environment} is the entity in which agents live, its state is changed by agents to achieve their goals. Although its behavior can be manipulated by users, this entity has only the goal of agents' synchronization, this means that the update of its state must be managed appropriately by users in multi-threaded systems, although immutable modeling helps to avoid critical runs. The environment encapsulates four important pieces of information: a set of agents living within the environment, the agent's mailboxes on which the others can send messages, the set of external actions visible at the environment level, and, finally, the component that returns the current state of the environment itself, in terms of belief.

\subsection{Middleware for Communication among Distributed Agents}

The current implementation of JaKtA focuses on managing multi-agent systems locally, limiting consideration to agents distributed over a network. To address this limitation and enable effective communication among distributed agents, the introduction of a dedicated middleware is proposed.
A middleware for communication among distributed agents would act as an intermediate layer between the agents themselves and the network, providing an abstraction that makes the complexities of distributed communication transparent to the programmer. The middleware aims to handle the sending and receiving of messages between agents located on different machines, ensuring information consistency.
The middleware would be responsible for establishing and maintaining connections between distributed agents. This includes managing communication protocols, handling lost connections, and addressing any delays or transmission errors.
Implementing middleware for communication among distributed agents would enhance JaKtA's ability to support scenarios where agents operate on heterogeneous or geographically distant networks. This would allow the creation of multi-agent systems that can collaborate and coordinate their actions on a larger scale, opening up new application possibilities for the JaKtA library.
