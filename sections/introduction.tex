\section{Introduzione}

\subsection{Agenti}

\subsubsection{Definizione}
Un agente è un'entità computazionale autonoma, dotata di capacità decisionali e di esecuzione indipendente. Gli agenti sono progettati per operare in un ambiente specifico, assumendo responsabilità e compiti definiti. La caratteristica chiave di un agente è la sua autonomia, che significa che può prendere decisioni in modo indipendente, senza richiedere intervento esterno per ogni azione intrapresa. Questa autonomia computazionale implica una proattività intrinseca, in cui gli agenti non attendono passivamente gli eventi ma agiscono attivamente per modificare lo stato del Sistema Multi-Agente (MAS).

Un MAS, o Sistema Multi-Agente, è un sistema computazionale composto da più agenti autonomi che interagiscono tra loro per raggiungere obiettivi comuni o individuali. In un MAS, ogni agente è un'entità autonoma, dotata di capacità di percezione, ragionamento, decisione e azione. Gli agenti in un MAS possono operare in modo indipendente e, al contempo, collaborare con gli altri agenti per affrontare problemi complessi o raggiungere obiettivi che possono essere difficili o impossibili da raggiungere per un singolo agente.

Un aspetto cruciale è la situazionalità degli agenti, il che significa che le loro azioni dipendono dal contesto in cui si trovano. Gli agenti, interagendo tra loro e con l'ambiente circostante, diventano entità intrinsecamente sociali. Questa socialità emerge poiché l'autonomia ha senso solo quando un agente è immerso in una società di agenti, evidenziando l'assenza di agenti veramente autonomi in isolamento. L'interazione avviene attraverso lo scambio di conoscenza e informazioni, mentre la proattività degli agenti li spinge a cambiare lo stato del MAS, influenzando altri agenti o l'ambiente. In questo contesto, gli agenti diventano componenti autonome, interattive e sociali che, mediante la loro capacità di autoregolamentazione, diventano fondamentali per la gestione dei sistemi complessi.

\subsubsection{Il Framework Belief-Desire-Intention}
Il framework BDI (Belief-Desire-Intention) è ampiamente accettato come uno dei più popolari e riusciti nel campo della tecnologia degli agenti. Definito da Rao e Georgeff, il framework si basa su tre concetti chiave: credenze (belief), desideri (desire) e intenzioni (intention). Gli agenti che seguono questo framework sono comunemente denominati agenti BDI.

\begin{itemize}
    \item \textbf{Credenze}: Per prendere decisioni, gli agenti devono avere una rappresentazione del mondo in cui vivono. Le credenze sono la conoscenza dell'ambiente che gli agenti raccolgono e possono memorizzare come parte della loro base di credenze.
    \item \textbf{Desideri}: I desideri rappresentano gli obiettivi principali del sistema. Gli agenti hanno a disposizione un insieme di piani che possono utilizzare per raggiungere i loro obiettivi, e la scelta del piano più adatto dipende dalla situazione corrente dell'agente e dell'ambiente.
    \item \textbf{Intenzioni}: Le intenzioni catturano la componente deliberativa del sistema. Rappresentano le azioni a cui l'agente si è impegnato per eseguire al fine di raggiungere i suoi obiettivi. Le intenzioni tengono traccia del progresso delle azioni e delle modifiche dell'ambiente durante l'esecuzione.
\end{itemize}

\subsection{JaKtA}
Panoramica di JaKtA e del suo funzionamento.

\subsection{The Necessity of a Distributed Middleware}
presentare la necessità di poter far comunicare i MAS in un contesto distribuito e elencare alcuni casi d'uso.
