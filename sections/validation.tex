\section{System Validation}
The JUnit testing framework in combination with the Ktor test engine was used to test the developed software modules.

Ktor provides a special \textit{test engine} that does not create a \textit{web server}, does not connect to the \textit{sockets}, and does not make any \textit{HTTP} requests. Instead, it connects directly to the internal mechanisms and directly processes an application call. This results in faster test execution than starting a full \textit{webserver}.

The Listing \ref{lst:BrokerBehaviorSpec} presents the test class \texttt{BrokerBehaviorSpec}, which tests the behavior of the \textit{broker}.

The test creates two simulated \textit{client}, \textit{fooMas} and \textit{barMas}, which are configured to support \textit{JSON} serialization and \textit{WebSockets}. Next, \textit{topic} (\texttt{fooTopic} and \texttt{barTopic}) and a test message (\texttt{message}) are established. Sessions are then created \textit{WebSocket} for subscription and publishing operations between the two \textit{clients}.

The test verifies the following behavior:

\begin{itemize}
    \item Obtains the list of available \textit{topic} through a \textit{HTTP GET} request.
    \item Verifies that the obtained \textit{topic} matches the previously established \texttt{fooTopic} and \texttt{barTopic}).
    \item Send a message from \texttt{fooMas} to \texttt{barTopic} through a subscription session, checking the error \texttt{BAD\_REQUEST} in response.
    \item Send a message from \texttt{fooMas} to \texttt{fooTopic} through a publish session and verify that \texttt{barSubscribeSession} receives the message.
    \item Send a message from \texttt{barMas} to \texttt{barTopic} through a publishing session and verify that \texttt{fooSubscribeSession} receives the message.
\end{itemize}

\begin{lstlisting}[caption={Test class \texttt{BrokerBehaviorSpec}.}, label={lst:BrokerBehaviorSpec}, language=Kotlin]
    testApplication {
        val fooMas = createClient {
            install(ContentNegotiation) {
                json()
            }
            install(WebSockets)
        }
        val barMas = createClient {
            install(ContentNegotiation) {
                json()
            }
            install(WebSockets)
        }
        val fooTopic = "fooTopic"
        val barTopic = "barTopic"
        val message = "testMessage"
        val fooSubscribeSession = fooMas.webSocketSession("/subscribe/$barTopic")
        val barSubscribeSession = barMas.webSocketSession("/subscribe/$fooTopic")
        val fooPublishSession = fooMas.webSocketSession("/publish/$fooTopic")
        val barPublishSession = barMas.webSocketSession("/publish/$barTopic")
        val availableTopics: Set<Topic> = fooMas.get("/topics").body()

        assertEquals(setOf(fooTopic, barTopic), availableTopics)

        fooSubscribeSession.send(Frame.Text(message))
        val response = fooSubscribeSession.incoming.receive()
        assertEquals(Error.BAD_REQUEST, Json.decodeFromString((response as Frame.Text).readText()))

        fooPublishSession.send(Frame.Text(message))
        val response2 = barSubscribeSession.incoming.receive()
        assertEquals(message, (response2 as Frame.Text).readText())

        barPublishSession.send(Frame.Text(message))
        val response3 = fooSubscribeSession.incoming.receive()
        assertEquals(message, (response3 as Frame.Text).readText())
    }
\end{lstlisting}
