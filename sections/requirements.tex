\section{Analisi dei Requisiti}

% Is there any implicit requirement hidden within this project's requirements?
%
% Is there any implicit hypothesis hidden within this project's requirements?
%
% Are there any non-functional requirements implied by this project's requirements?

% What model / paradigm / technology is the best suited to face this project's requirements?
%
% What's the abstraction gap among the available models / paradigms / technologies and the problem to be solved?

\subsection{Obiettivi}

Il goal principale del progetto è quello di estendere la libreria JaKtA per permettere la costruzione di Sistemi Multi-Agente distribuiti in rete evitando all'utilizzatore della libreria di gestire le complessità delle comunicazioni di rete.

\subsection{Requisiti}
Sulla base dell'obiettivo principale del progetto, sono stati individuati i seguenti requisiti funzionali, non funzionali e di implementazione.

\subsubsection{Requisiti Funzionali}
I requisiti funzionali delineano le funzioni specifiche del sistema, cioè cosa il sistema deve fare:

\begin{enumerate}
      \item Gestione della distribuzione degli agenti:
            \begin{enumerate}
                  \item  Il \textit{middleware} deve supportare la distribuzione degli agenti su macchine eterogenee,
                        consentendo la loro esecuzione coordinata su diverse posizioni geografiche.
            \end{enumerate}

      \item Implementazione di un meccanismo di \textit{dispatch} dei messaggi tra i vari agenti:
            \begin{enumerate}
                  \item Il \textit{middleware} deve sfruttare un \textit{broker} per la comunicazione in rete:
                        \begin{enumerate}
                              \item Il \textit{broker} deve consentire l'aggiunta di \textit{publisher} associati a un determinato \textit{topic}.
                              \item Il \textit{broker} deve permettere la rimozione di \textit{publisher} associati a un \textit{topic} specifico.
                              \item Il \textit{broker} deve supportare l'aggiunta di \textit{subscriber} associati a un determinato \textit{topic}.
                              \item Il \textit{broker} deve permettere la rimozione di \textit{subscriber} associati a un \textit{topic} specifico.
                              \item Il broker deve fornire l'elenco di tutti i topic disponibili.
                        \end{enumerate}
                  \item Il \textit{middleware} deve incorporare un \textit{client} che si interfacci con il \textit{broker}:
                        \begin{enumerate}
                              \item Il \textit{client} deve fornire la funzionalità di pubblicazione di un messaggio su un determinato \textit{topic}.
                              \item Il client deve offrire la possibilità di effettuare la sottoscrizione a un \textit{topic} specifico.
                              \item Il sistema deve consentire al \textit{client} di effettuare una trasmissione \textit{broadcast}, inviando un messaggio a tutti gli altri \textit{client} connessi.
                              \item  Il \textit{client} deve essere in grado di stabilire e gestire connessioni \textit{WebSocket} verso un \textit{broker} specificato tramite indirizzo \textit{IP} e porta.
                        \end{enumerate}
            \end{enumerate}
      \item  Comunicazione Distribuita e Coordinazione:
            \begin{enumerate}
                  \item  Il \textit{middleware} deve gestire la comunicazione distribuita tra agenti, permettendo lo scambio di messaggi e la coordinazione delle attività.
                  \item Deve supportare la sincronizzazione degli agenti distribuiti per garantire un'interazione coesa e coordinata.
            \end{enumerate}
\end{enumerate}

\subsubsection{Requisiti non Funzionali}
I requisiti non funzionali definiscono attributi di qualità, vincoli e proprietà generali del sistema:

\begin{enumerate}
      \item Il \textit{middleware} deve essere realizzato in moduli separati, in modo da poter essere facilmente manutenibile ed estendibile.
      \item  Il \textit{middleware} deve essere progettato per scalare orizzontalmente, supportando facilmente l'aggiunta di nuovi nodi senza degradare le prestazioni.
\end{enumerate}

\subsubsection{Requisiti di Implementazione}
I requisiti di implementazione riguardano gli aspetti tecnologici e metodologici dell'implementazione del sistema:

\begin{enumerate}
      \item  Il progetto deve essere implementato utilizzando il linguaggio di programmazione Kotlin per garantire coerenza e compatibilità con la libreria JaKtA.
      \item Utilizzare il \textit{framework} Ktor per implementare il \textit{middleware} per l'interazione tra Sistemi Multi-Agente distribuiti.
\end{enumerate}

% \subsubsection{Requisiti non funzionali}

% \begin{enumerate}
%     \item  Il sistema deve essere scalabile, consentendo l'aggiunta di nuovi client e broker senza compromettere le prestazioni generali del sistema.
%     \item
%     \item
% \end{enumerate}