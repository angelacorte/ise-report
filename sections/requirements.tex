\section{Analisi dei Requisiti}

% Is there any implicit requirement hidden within this project's requirements?
%
% Is there any implicit hypothesis hidden within this project's requirements?
%
% Are there any non-functional requirements implied by this project's requirements?

% What model / paradigm / technology is the best suited to face this project's requirements?
%
% What's the abstraction gap among the available models / paradigms / technologies and the problem to be solved?

\subsection{Obiettivi}

Il goal principale del progetto è quello di estendere la libreria JaKtA e il relativo framework per la costruzione di Sistemi Multi-Agenti,
con la creazione di un middleware per l'interazione tra sistemi multi-agente distribuiti.

\subsection{Requisiti}
A partire dal goal principale del progetto, sono stati individuati diversi requisiti sistema in base alle diverse categorie di requisiti identificate:

\subsubsection{Requisiti Funzionali}
\begin{enumerate}
      \item Gestione della distribuzione degli agenti:
            \begin{enumerate}
                  \item  Il middleware deve supportare la distribuzione degli agenti su macchine eterogenee,
                        consentendo la loro esecuzione coordinata su diverse posizioni geografiche.
            \end{enumerate}

      \item Implementazione di un meccanismo di dispatch dei messaggi tra i vari agenti:
            \begin{enumerate}
                  \item Il middleware deve essere composto da un broker:
                        \begin{enumerate}
                              \item Il broker deve consentire l'aggiunta di publisher associati a un determinato topic.
                              \item Il broker deve permettere la rimozione di publisher associati a un topic specifico.
                              \item Il broker deve supportare l'aggiunta di subscriber associati a un determinato topic.
                              \item Il broker deve permettere la rimozione di subscriber associati a un topic specifico.
                              \item Il broker deve fornire l'elenco di tutti i topic disponibili.
                              \item Il sistema deve utilizzare strutture dati sincronizzate per gestire in modo sicuro e concorrente l'associazione di publishers e subscribers ai topic.
                        \end{enumerate}
                  \item Il middleware deve essere composto da un client:
                        \begin{enumerate}
                              \item Il client deve fornire la funzionalità di pubblicare un messaggio su un determinato topic.
                              \item Il client deve offrire la possibilità di sottoscriversi a un topic specifico.
                              \item Il sistema deve consentire al client di effettuare una trasmissione broadcast, inviando un messaggio a tutti i client connessi.
                              \item  Il client deve essere in grado di stabilire e gestire connessioni WebSocket verso un server specificato tramite host e porta.
                        \end{enumerate}
            \end{enumerate}

      \item  Comunicazione Distribuita e Coordinazione
            \begin{enumerate}
                  \item  Il middleware deve gestire la comunicazione distribuita tra agenti, facilitando
                        lo scambio di messaggi e la coordinazione delle attività.
                  \item Deve supportare la sincronizzazione degli agenti distribuiti per garantire
                        un'interazione coesa e coordinata.
            \end{enumerate}
\end{enumerate}

\subsubsection{Requisiti Non Funzionali}

\begin{enumerate}
      \item Il progetto deve essere realizzato in moduli separati, in modo da poter essere facilmente estendibile.
      \item  Il middleware deve essere progettato per scalare orizzontalmente, supportando facilmente l'aggiunta di nuovi nodi senza degradare le prestazioni.
\end{enumerate}

\subsubsection{Requisiti di Implementazione}

\begin{enumerate}
      \item  Il progetto deve essere implementato utilizzando il linguaggio di programmazione
            Kotlin per garantire coerenza e compatibilità con la libreria JaKtA.
      \item Utilizzare il framework Ktor per implementare il middleware per l'interazione tra
            sistemi multi-agente distribuiti.
\end{enumerate}

% \subsubsection{Requisiti non funzionali}

% \begin{enumerate}
%     \item  Il sistema deve essere scalabile, consentendo l'aggiunta di nuovi client e broker senza compromettere le prestazioni generali del sistema.
%     \item
%     \item
% \end{enumerate}