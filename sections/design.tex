\section{Design}

%This is where the logical / abstract contribution of the project is presented.

%Notice that, when describing a software project, three dimensions need to be taken into account: structure, behaviour, and interaction.

%Always remember to report \textbf{why} a particular design has been chosen.
%Reporting wrong design choices which has been evalued during the design phase is welcome too.

\subsection{Structure}

È stato deciso di adottare una struttura tale da dividere le funzionalità del broker da quelle del client.

\subsubsection{Client}

L'applicativo del client si occupa di gestire la comunicazione attraverso WebSocket utilizzando il framework Ktor in Kotlin. \\

Per affrontare la natura distribuita del progetto, il sistema è strutturato attorno a entità chiave: Client,
 WebSocketSession e i modelli di dati per la comunicazione (SerializableSendMessage e SerializableBroadcastMessage). \\
 
 L'interfaccia Client funge da punto principale di accesso, incapsulando le funzionalità necessarie per la comunicazione. \\
 
 La classe WebSocketsClient è un componente chiave che gestisce le connessioni WebSocket. \\
 
 I modelli di dati per la comunicazione rappresentano i messaggi scambiati tra il client e il server. \\

Vengono implementate tre funzionalità principali:
\begin{itemize}
    \item \texttt{publish}: si occcupa di inviare un messaggio ad un determinato 'topic' attraverso una connessione WebSocket;
    \item \texttt{broadcast}: simile alla funzione sopra citata ma è destinata ad unviare un messaggio a tutti i client connessi;
    \item \texttt{subscribe}: gestisce l'iscrizione del client ad un particolare 'topic' attraverso una connessione WebSocket.
\end{itemize}

Per quanto riguarda la gestione dei messaggi in arrivo, la funzione \texttt{subscribe} rimane in ascolto dei messaggi in arrivo sulla sessione WebSocket
associata al 'topic', i messaggi ricevuti vengono dunque deserializzati e memorizzati.


%Which entities need to by modelled to solve the problem? 
%
(UML Class diagram)

%How should entities be modularised?
%
(UML Component / Package / Deployment Diagrams)

\subsection{Behaviour}

How should each entity behave?
%
(UML State diagram or Activity Diagram)

\subsection{Interaction}

How should entities interact with each other?
%
(UML Sequence Diagram)
