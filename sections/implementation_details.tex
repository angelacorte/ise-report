\section{Implementation Details}

\subsection{Technologies}

\subsubsection{Kotlin}
Kotlin\cite{kotlin} was chosen as the programming language for the development of the project, mainly because the
codebase in JaKtA is also written in Kotlin.

Kotlin is a modern and highly expressive programming language designed to work seamlessly with Java\cite{java}.
With clear and concise syntax, advanced nullability handling, function extensions, and native support for object-oriented programming, Kotlin provides a robust
Kotlin provides a robust and flexible development environment.

\subsubsection{Gradle}
The \textit{build system} chosen was the one already present in JaKtA, namely Gradle\cite{gradle}.

Gradle is a build automation and dependency management system. Using a declarative configuration model, Gradle
simplifies the process of creating and managing projects by providing a clear and powerful syntax based on Groovy\cite{groovy} or Kotlin.

\subsubsection{Ktor}
For the management of network communications between \textit{client} and \textit{broker}, the Ktor{cite{ktor}
library was chosen.

Ktor is a modern, lightweight Web development \textit{framework} written in Kotlin, designed to simplify the creation of
robust, high-performance Web applications. It offers a declarative and concise approach to handling \textit{HTTP}
requests and building web services. The \textit{framework} allows asynchronous and non-blocking programming, plus it has
the ability to integrate effortlessly with other Kotlin technologies, such as \textit{coroutines}

\subsubsection{JUnit}
The \textit{framework} JUnit{cite{junit} was used to test the operation of the developed modules.

Created to simplify and improve the process of \textit{testing}, JUnit provides a set of \textit{annotation} and
\textit{assert} that allow developers to easily define and execute \textit{unit tests}.

\subsection{Broker}
The \textit{broker} implemented in Kotlin uses the Ktor \textit{framework} to handle asynchronous communication through \textit{WebSocket} and \textit{HTTP} requests. This distributed \textit{broker} is based on two main components: \textit{Cache} and \textit{SubscriptionManager}.

Interface \texttt{Cache<T>}:

\begin{itemize}
    \item It is responsible for the temporary storage of data associated with specific \textit{topic}.
    \item Provides operations such as logging data, releasing resources, and reading data associated with a specific \textit{topic}.
\end{itemize}

\texttt{SubscriptionManager<T>}:

\begin{itemize}
    \item Provides operations to add and remove \textit{publisher} and \textit{subscriber} associated with specific \textit{topic}.
    \item Provides information about available \textit{topic} and sections associated with a specific \textit{topic}.
\end{itemize}

Endpoint and Routing:

\begin{itemize}
    \item The module \texttt{configureWebSockets} defines the \textit{WebSocket endpoints} for publishing and subscribing
    \item to data related to specific \textit{topic}.
    \item The \textit{configureRouting} module configures the \textit{route HTTP} to obtain information about the available \textit{topic}.
\end{itemize}

\subsection{Adapter}
To realize a multi-agent system version that can communicate with the outside world, the \texttt{Mas} interface was extended with the \texttt{Dmas} interface:

\begin{lstlisting}[language=Kotlin]
interface Dmas : Mas {
    val network: Network
    val remoteServices: List<RemoteService>
}
\end{lstlisting}

An implementation of the \texttt{Dmas} can thus be used in place of the \texttt{Mas} by the main application logic of the \textit{framework}, but in addition it can make use of the \texttt{Network} and \texttt{RemoteService} to
implement the external communication logic.

As for the implementation of the \texttt{Network} interface, which encapsulates the communication logic between \texttt{Dmas} across the network, a solution based on the use of
of a \textit{client} \textit{WebSocket}, which will be discussed in detail in the following section.

\subsection{Client}
The \textit{client} application is designed to handle communication through \textit{WebSocket},
making use of the \textit{framework} Ktor in Kotlin. To address the distributed nature of the project,
key entities were defined, including \texttt{Client}, \texttt{WebSocketSession}, and data models
\texttt{SerializableSendMessage} and \\ \texttt{SerializableBroadcastMessage}.

The \texttt{Client} interface plays the role of the main access point, encapsulating the core functionality for
communication. Its key implementation is represented by the \texttt{WebSocketsClient} class.

The \texttt{WebSocketsClient} class is a central component for managing \textit{WebSocket} connections.
It is responsible for creating and managing \textit{WebSocket} sessions for publishing,
subscription and transmission of messages. It also keeps track of active connections, disconnections, and incoming messages.

The models \texttt{SerializableSendMessage} and \texttt{SerializableBroadcastMessage} are essential to represent the messages exchanged between the \textit{client} and
the \textit{server}. Their structure is designed to ensure proper serialization and deserialization of data.

Three main features are implemented:
\begin{itemize}
    \item \texttt{publish}: deals with sending a message to a given \textit{topic} through a \textit{WebSocket} connection;
    \item \texttt{broadcast}: similar to the above function but is intended to send a message to all connected \textit{clients};
    \item \texttt{subscribe}: handles the subscription of the \textit{client} to a particular \textit{topic} through a \textit{WebSocket} connection.
\end{itemize}
Regarding the handling of incoming messages, there is the \textt{subscribe} function.
This function listens for messages sent to the specified \textit{topic}, deserializes the received messages, and stores them for later use.
This mechanism is critical to ensure effective two-way communication between the \textit{client} and the \textit{server}.

\subsection{Implementation Notes}
This section seeks to shed light on some implementation choices that may be considered suboptimal, which were made for reasons of time and/or complexity.

\subsection{RemoteService}
The \texttt{RemoteService} serves as a representation of a remote agent, i.e., an agent that belongs to a different \texttt{Dmas} than the current one, but can be reached through the network.
It is identified by its name and is specified when creating the \texttt{Dmas}. It is important to note that, at present, no mechanism has been implemented to make
unique \texttt{RemoteService} names for all \texttt{Dmas}, so it is possible for two different \texttt{Dmas} to have a \texttt{RemoteService} with the same name.

The option of prefixing the name of the \texttt{RemoteService} with a unique code was evaluated, however, since in the current design the \texttt{RemoteService} is specified at the time of creation
of the \texttt{Dmas}, it is not possible to know a priori the unique code to be used. There is also a risk that two different \texttt{Dmas}, being in separate processes will generate the same code
prefix.

So, as a temporary solution, a control mechanism has been implemented at the broker level that prevents two different \texttt{Dmas} from publishing messages on a topic (corresponding to the remote service) with the same name.
Design notes related to this mechanism can be found in the section:~\ref{sec:publishers-same-name}.

\subsubsection{Reserved names}.
There are two \texttt{RemoteService} names that are to be considered reserved: \texttt{broker} and \texttt{broadcast}. The former is the name of the \texttt{RemoteService} of the Broker, while the latter is used
to perform broadcasting of messages.

Currently, there is no mechanism in place to prevent a programmer from calling an agent of their \texttt{Dmas} with the above names, however, doing so could cause behaviors
unexpected system behavior.