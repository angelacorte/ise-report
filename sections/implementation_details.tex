\section{Implementation Details}

\subsection{Client}
L'applicativo del client è progettato per gestire la comunicazione attraverso WebSocket,
facendo uso del framework Ktor in Kotlin. Per affrontare la natura distribuita del progetto,
sono state definite entità chiave, tra cui \textit{Client}, \textit{WebSocketSession}, e i modelli di dati
\textit{SerializableSendMessage} e \textit{SerializableBroadcastMessage}. \\

L'interfaccia Client svolge il ruolo di punto principale di accesso, incapsulando le funzionalità fondamentali per la
comunicazione. La sua implementazione chiave è rappresentata dalla classe \textit{WebSocketsClient}.\\

La classe WebSocketsClient è una componente centrale per gestire le connessioni WebSocket.
È responsabile della creazione e gestione di sessioni WebSocket per la pubblicazione,
la sottoscrizione e la trasmissione di messaggi. Inoltre, tiene traccia delle connessioni attive, delle disconnessioni e dei messaggi in arrivo.\\

I modelli \textit{SerializableSendMessage} e \textit{SerializableBroadcastMessage} sono essenziali per rappresentare i messaggi scambiati tra il client e
il server. La loro struttura è progettata per garantire una corretta serializzazione e deserializzazione dei dati. \\

Vengono implementate tre funzionalità principali:
\begin{itemize}
    \item \texttt{publish}: si occcupa di inviare un messaggio ad un determinato 'topic' attraverso una connessione WebSocket;
    \item \texttt{broadcast}: simile alla funzione sopra citata ma è destinata ad unviare un messaggio a tutti i client connessi;
    \item \texttt{subscribe}: gestisce l'iscrizione del client ad un particolare 'topic' attraverso una connessione WebSocket.
\end{itemize}

Per quanto riguarda la gestione dei messaggi in arrivo, la funzione \texttt{subscribe}.
Questa funzione rimane in ascolto dei messaggi inviati al 'topic' specificato, deserializza i messaggi ricevuti e li memorizza per un successivo utilizzo.
Questo meccanismo è fondamentale per garantire una comunicazione bidirezionale efficace tra il client e il server.

