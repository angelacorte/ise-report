\section{Dettagli Implementativi}

\subsection{Tecnologie Utilizzate}

\subsubsection{Kotlin}

Per lo sviluppo del progetto è stato scelto di utilizzare Kotlin\cite{kotlin} come linguaggio di programmazione, questo perché la base di codice presente in JaKtA è anch'essa scritta in tale linguaggio.

Kotlin è un linguaggio di programmazione moderno e altamente espressivo, progettato per interoperare completamente con Java\cite{java}.
Con una sintassi chiara e concisa, gestione avanzata delle nullità, estensioni delle funzioni e il supporto nativo per la programmazione orientata agli oggetti, Kotlin offre un ambiente di sviluppo robusto e flessibile.

\subsubsection{Gradle}

Come \textit{build system} è stato scelto di utilizzare quello già presente in JaKtA, ovvero Gradle\cite{gradle}.

Gradle è un sistema di automazione della compilazione e gestione delle dipendenze. Basato su un modello di configurazione dichiarativa, Gradle semplifica il processo di costruzione e gestione dei progetti, fornendo una sintassi chiara e potente basata su Groovy\cite{groovy} o Kotlin.

\subsubsection{Ktor}

Per la gestione delle comunicazioni di rete tra \textit{client} e \textit{broker} è stato scelto di utilizzare la libreria Ktor\cite{ktor}.

Ktor è un \textit{framework} di sviluppo web moderno e leggero scritto in Kotlin, progettato per semplificare la creazione di applicazioni web robuste e performanti. Offre un approccio dichiarativo e conciso per gestire le richieste \textit{HTTP} e costruire servizi web. Il \textit{framework} permette la programmazione asincrona e non bloccante, inoltre ha la capacità di integrarsi senza sforzo con altre tecnologie Kotlin, come le \textit{coroutine}

\subsubsection{JUnit}

Per testare il funzionamento dei moduli sviluppati è stato utilizzato il \textit{framework} JUnit\cite{junit}.

Creato per semplificare e migliorare il processo di \textit{testing}, JUnit fornisce un set di \textit{annotation} e \textit{assert} che consentono agli sviluppatori di definire e eseguire facilmente gli \textit{unit test}.

\subsection{Broker}
Il \textit{broker} implementato in Kotlin utilizza il \textit{framework} Ktor per gestire la comunicazione asincrona attraverso \textit{WebSocket} e le richieste \textit{HTTP}. Questo \textit{broker} distribuito si basa su due principali componenti: \texttt{Cache} e \texttt{SubscriptionManager}.

Interfaccia \texttt{Cache<T>}:

\begin{itemize}
    \item È Responsabile della memorizzazione temporanea di dati associati a specifici \textit{topic}.
    \item Fornisce operazioni come la registrazione di dati, la liberazione di risorse e la lettura di dati associati a un \textit{topic} specifico.
\end{itemize}

\texttt{SubscriptionManager<T>}:

\begin{itemize}
    \item Offre operazioni per aggiungere e rimuovere \textit{publisher} e \textit{subscriber} associati a specifici \textit{topic}.
    \item Fornisce informazioni sui \textit{topic} disponibili e sulle sezioni associate a un determinato \textit{topic}.
\end{itemize}

Endpoint e Routing:

\begin{itemize}
    \item Il modulo \texttt{configureWebSockets} definisce i \textit{WebSocket endpoint} per la pubblicazione e la sottoscrizione di dati relativi a specifici \textit{topic}.
    \item Il modulo \textit{configureRouting} configura le \textit{route HTTP} per ottenere informazioni sui \textit{topic} disponibili.
\end{itemize}

\subsection{Adapter}
Per realizzare una versione di sistema multi-agente in grado di comunicare con l'esterno, è stata estesa l'interfaccia \texttt{Mas} con l'interfaccia \texttt{Dmas}:

\begin{lstlisting}[language=Kotlin]
interface Dmas : Mas {
    val network: Network
    val remoteServices: List<RemoteService>
}
\end{lstlisting}

Un'implementazione del \texttt{Dmas} può dunque essere usata in vece del \texttt{Mas} dalla logica applicativa principale del \textit{framework}, ma in aggiunta può servirsi di \texttt{Network} e \texttt{RemoteService} per
implementare la logica di comunicazione con l'esterno.

Per quanto riguarda l'implementazione dell'interfaccia \texttt{Network}, che incapsula la logica di comunicazione tra \texttt{Dmas} attraverso la rete, è stata scelta una soluzione basata sull'uso
di un \textit{client} \textit{WebSocket}, che verrà trattato nel dettaglio nel paragrafo successivo.

\subsection{Client}
L'applicativo del \textit{client} è progettato per gestire la comunicazione attraverso \textit{WebSocket},
facendo uso del \textit{framework} Ktor in Kotlin. Per affrontare la natura distribuita del progetto,
sono state definite entità chiave, tra cui \texttt{Client}, \texttt{WebSocketSession}, e i modelli di dati
\texttt{SerializableSendMessage} e \texttt{SerializableBroadcastMessage}.

L'interfaccia \texttt{Client} svolge il ruolo di punto principale di accesso, incapsulando le funzionalità fondamentali per la
comunicazione. La sua implementazione chiave è rappresentata dalla classe \texttt{WebSocketsClient}.

La classe \texttt{WebSocketsClient} è una componente centrale per gestire le connessioni \textit{WebSocket}.
È responsabile della creazione e gestione di sessioni \textit{WebSocket} per la pubblicazione,
la sottoscrizione e la trasmissione di messaggi. Inoltre, tiene traccia delle connessioni attive, delle disconnessioni e dei messaggi in arrivo.

I modelli \texttt{SerializableSendMessage} e \texttt{SerializableBroadcastMessage} sono essenziali per rappresentare i messaggi scambiati tra il \textit{client} e
il  \textit{server}. La loro struttura è progettata per garantire una corretta serializzazione e deserializzazione dei dati.

Vengono implementate tre funzionalità principali:
\begin{itemize}
    \item \texttt{publish}: si occupa di inviare un messaggio ad un determinato \textit{topic} attraverso una connessione \textit{WebSocket};
    \item \texttt{broadcast}: simile alla funzione sopra citata ma è destinata ad inviare un messaggio a tutti i \textit{client} connessi;
    \item \texttt{subscribe}: gestisce l'iscrizione del \textit{client} ad un particolare \textit{topic} attraverso una connessione \textit{WebSocket}.
\end{itemize}

Per quanto riguarda la gestione dei messaggi in arrivo, è presente la funzione \texttt{subscribe}.
Questa funzione rimane in ascolto dei messaggi inviati al \textit{topic} specificato, deserializza i messaggi ricevuti e li memorizza per un successivo utilizzo.
Questo meccanismo è fondamentale per garantire una comunicazione bidirezionale efficace tra il \textit{client} e il  \textit{server}.

\subsection{Note Implementative}
In questa sezione si vuole fare chiarezza su alcune scelte implementative che possono ritenersi sub ottimali, che sono state fatte per motivi di tempo e/o di complessità.

\subsubsection{RemoteService}
Il \texttt{RemoteService} funge da rappresentazione di un agente remoto, cioè un agente che appartiene ad un \texttt{Dmas} diverso da quello corrente, ma raggiungibile attraverso la rete.
Esso è identificato dal suo nome e viene specificato in fase di creazione del \texttt{Dmas}. È importante notare che, al momento, non è stato implementato un meccanismo che renda
univoci i nomi dei \texttt{RemoteService} per tutti i \texttt{Dmas}, quindi è possibile che due \texttt{Dmas} diversi abbiano un \texttt{RemoteService} con lo stesso nome.

È stata valutata l'opzione di prefiggere al nome del \texttt{RemoteService} un codice univoco, tuttavia, siccome nel corrente design il \texttt{RemoteService} viene specificato in fase di creazione
del \texttt{Dmas}, non è possibile conoscere a priori il codice univoco da utilizzare. C'è inoltre il riscio che due \texttt{Dmas} diversi, essendo in processi separati, generino lo stesso codice
prefisso.

Dunque, come soluzione temporanea, è stato implementato un meccanismo di controllo a livello del broker che impedisce che due \texttt{Dmas} diversi pubblichino messaggi su un topic (corrispondente al remote service) con lo stesso nome.
Le note di design relative a questo meccanismo sono presenti nella sezione: \ref{sec:publishers-same-name}.

\subsubsection{Nomi riservati}
Esistono due nomi di \texttt{RemoteService} da ritenersi riservati: \texttt{broker} e \texttt{broadcast}. Il primo è il nome del \texttt{RemoteService} del Broker, mentre il secondo è utilizzato
per effettuare broadcasting dei messaggi.

Attualmente non è in funzione alcun meccanismo che impedisca un programmatore di chiamare un agente del proprio \texttt{Dmas} con i nomi sopra citati, tuttavia, farlo potrebbe causare comportamenti
inaspettati del sistema.