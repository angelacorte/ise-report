\section{Usage Examples}

To showcase the actual usefulness of the introduced middleware, some examples have been developed; among these, one of the most significant is \textit{Telephone Game}\ref{lst:TelephoneGame}. The example implements the following behavior:

\begin{itemize}
    \item there are N agents, distributed among N \textit{MAS};
    \item each agent is assigned an ID, ranging from 0 to N - 1;
    \item each agent waits for a message from the agent with id \(ID + N - 1\);
    \item when a message is received, the agent sends it to the agent with id \((ID + 1) \% N\);
    \item the program ends when there has been a fixed amount of messages sent.
\end{itemize}

\begin{lstlisting}[caption={Usage example: \texttt{TelephoneGame}.}, label={lst:TelephoneGame}, language=Kotlin]
    fun main(args: Array<String>) {
        val host: String = args[0]
        val port: Int = args[1].toInt()
        val id: Int = args[2].toInt()
        val nAgents: Int = args[3].toInt()
        val nMessages = nAgents * 2
        dmas {
            environment {
                actions {
                    action(send, 2) {
                        val receiver: Atom = argument(0)
                        val message: Struct = argument(1)
                        this.sendMessage(receiver.value, Message(this.sender, Tell, message))
                    }
                }
            }
    
            agent("ag$id") {
                beliefs {
                    fact { receiver("ag${(id + 1) % nAgents}") }
                }
                if (id == 0) goals { achieve(start) }
                plans {
                    +achieve(start) onlyIf { receiver(source(self), R) } then {
                        execute(print("Starting..."))
                        execute(send(R, count(1)))
                    }
                    +count(source(`_`), X) onlyIf {
                        (X lowerThan nMessages) and (M `is` (X + 1)) and receiver(source(self), R)
                    } then {
                        execute(print("Sending ", M))
                        execute(send(R, count(M)))
                    }
                    +count(source(`_`), X) onlyIf { X greaterThanOrEqualsTo nMessages } then {
                        execute(print("Done!"))
                        execute("stop")
                    }
                }
            }
    
            service {
                name("ag${(id + nAgents - 1) % nAgents}")
            }
    
            network {
                websocketNetwork(host, port)
            }
        }.start()
    }
\end{lstlisting}

\subsection{Broadcast}
The next example showcases a different situation:
\begin{itemize}
    \item there is a Broadcaster agent, executing in his own \textit{MAS};
    \item there are 2 Receiver agents, both executing in the same \textit{MAS} which is different from the Broadcaster's;
    \item the Broadcaster broadcasts a message to the system;
    \item the receivers receive the message and print it.
\end{itemize}

\begin{lstlisting}[caption={Usage example: \texttt{Broadcast}: Broadcaster's code.}, label={lst:Broadcast}, language=Kotlin]
    fun main() {
        val broadcastAction = object : AbstractExternalAction("broadcast", 2) {
            override fun action(request: ExternalRequest) {
                val type = request.arguments[0].castToAtom()
                val message = request.arguments[1].castToStruct()
                when (type.value) {
                    "tell" -> broadcastMessage(Message(request.sender, Tell, message))
                    "achieve" -> broadcastMessage(
                        Message(request.sender, Achieve, message),
                    )
                }
            }
        }
    
        val env = Environment.of(
            externalActions = mapOf(
                broadcastAction.signature.name to broadcastAction,
            ),
        )
    
        val broadcaster = Agent.of(
            name = "broadcaster",
            events = listOf(
                Event.ofAchievementGoalInvocation(
                    it.unibo.jakta.agents.bdi.goals.Achieve.of(
                        Jakta.parseStruct("broadcast"),
                    ),
                ),
            ),
            planLibrary = PlanLibrary.of(
                Plan.ofAchievementGoalInvocation(
                    value = Jakta.parseStruct("broadcast"),
                    goals = listOf(
                        ActInternally.of(Jakta.parseStruct("print(\"Broadcast message\")")),
                        Act.of(Jakta.parseStruct("broadcast(tell, greetings)")),
                    ),
                ),
            ),
        )
    
        DMas.fromWebSocketNetwork(
            ExecutionStrategy.oneThreadPerAgent(),
            env,
            listOf(broadcaster),
            emptyList(),
            "localhost",
            8080,
        ).start()
    }
\end{lstlisting}

\begin{lstlisting}[caption={Usage example: \texttt{Broadcast}: Receiver's code.}, label={lst:Broadcast}, language=Kotlin]
    fun main() {
        val env = Environment.of()
    
        val alice = Agent.of(
            name = "alice",
            planLibrary = PlanLibrary.of(
                Plan.ofBeliefBaseAddition(
                    belief = Belief.from(Jakta.parseStruct("greetings(source(Sender))")),
                    goals = listOf(
                        ActInternally.of(Jakta.parseStruct("print(\"Received message from: \", Sender)")),
                    ),
                ),
            ),
        )
    
        val bob = Agent.of(
            name = "bob",
            planLibrary = PlanLibrary.of(
                Plan.ofBeliefBaseAddition(
                    belief = Belief.from(Jakta.parseStruct("greetings(source(Sender))")),
                    goals = listOf(
                        ActInternally.of(Jakta.parseStruct("print(\"Received message from: \", Sender)")),
                    ),
                ),
            ),
        )
    
        DMas.withEmbeddedBroker(ExecutionStrategy.oneThreadPerMas(), env, listOf(alice, bob), emptyList()).start()
    }

\end{lstlisting}