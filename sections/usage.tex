\section{Esempi di Utilizzo}

Per dimostrare l'effettiva utilità del \textit{middleware} introdotto in JaKtA, sono stati sviluppati degli esempi; tra questi, uno dei più significativi è \textit{Telephone Game}\ref{lst:TelephoneGame}. L'esempio implementa il seguente comportamento:

\begin{itemize}
    \item Sono presenti N agenti, distribuiti in N \textit{MAS};
    \item Ogni agente ha un ID univoco;
    \item Ogni agente deve attende il messaggio dall'agente \(ID + N - 1\);
    \item Quando l'agente riceve il messaggio lo replica verso l'agente \((ID + 1) \% N\);
    \item Il programma termina quando gli agenti hanno scambiato il numero prestabilito di messaggi.
\end{itemize}

\begin{lstlisting}[caption={Esempio di utilizzo \texttt{TelephoneGame}.}, label={lst:TelephoneGame}, language=Kotlin]
    fun main(args: Array<String>) {
        val host: String = args[0]
        val port: Int = args[1].toInt()
        val id: Int = args[2].toInt()
        val nAgents: Int = args[3].toInt()
        val nMessages = nAgents * 2
        dmas {
            environment {
                actions {
                    action(send, 2) {
                        val receiver: Atom = argument(0)
                        val message: Struct = argument(1)
                        this.sendMessage(receiver.value, Message(this.sender, Tell, message))
                    }
                }
            }
    
            agent("ag$id") {
                beliefs {
                    fact { receiver("ag${(id + 1) % nAgents}") }
                }
                if (id == 0) goals { achieve(start) }
                plans {
                    +achieve(start) onlyIf { receiver(source(self), R) } then {
                        execute(print("Starting..."))
                        execute(send(R, count(1)))
                    }
                    +count(source(`_`), X) onlyIf {
                        (X lowerThan nMessages) and (M `is` (X + 1)) and receiver(source(self), R)
                    } then {
                        execute(print("Sending ", M))
                        execute(send(R, count(M)))
                    }
                    +count(source(`_`), X) onlyIf { X greaterThanOrEqualsTo nMessages } then {
                        execute(print("Done!"))
                        execute("stop")
                    }
                }
            }
    
            service {
                name("ag${(id + nAgents - 1) % nAgents}")
            }
    
            network {
                websocketNetwork(host, port)
            }
        }.start()
    }
\end{lstlisting}

Per una panoramica completa delle funzionalità del \textit{framework}, si rimanda al \textit{repository} ufficiale \cite{repo}, dove sono presenti diversi esempi di utilizzo.